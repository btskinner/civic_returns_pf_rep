%%%%%%%%%%%%%%%%%%%%%%%%%%%%%%%%%%%%%%%%%%%%%%%%%%%%%%%%%%%%%%%%%%%%%%%%%%%%%%%%
%
% [ PROJ ] Civic returns: propensity forests
% [ FILE ] paper_tabfig.tex
% [ AUTH ] Benjamin Skinner & Will Doyle
% [ INIT ] 10 July 2019
%
%%%%%%%%%%%%%%%%%%%%%%%%%%%%%%%%%%%%%%%%%%%%%%%%%%%%%%%%%%%%%%%%%%%%%%%%%%%%%%%%

% ------------------------------------------------------------------------------
% SETTINGS
% ------------------------------------------------------------------------------

% class and font size
\documentclass[12pt]{article}

% packages
\usepackage[margin=1in]{geometry} % margins
\usepackage[T1]{fontenc}          % fonts
\usepackage[utf8]{inputenc}
\usepackage{amstext}
\usepackage{amssymb}
\usepackage{amsmath}
\usepackage{array}
\usepackage{mathptmx}
\usepackage[style=apa,          % bibliography
backend=biber,
sortcites=true,
sorting=nyt,
language=american]{biblatex}
\usepackage{tabularx}           % tables
\usepackage{booktabs}
\usepackage{rotating}
\usepackage{longtable}
\usepackage{dcolumn}
\usepackage{rotating}
\usepackage{caption}
\usepackage{graphicx}           % graphics
\usepackage[linktocpage]{hyperref} % links
\usepackage{setspace}           % spacing options
\usepackage{fancyhdr}           % header & footer options
\usepackage{datetime}           % better date
\usepackage{enumitem}           % better lists
\usepackage{appendix}           % appendix
\usepackage[nolists,            % endfloat tables & figures
nomarkers]{endfloat}

% directories
\newcommand{\rootpath}{..}
\newcommand{\figdir}{\rootpath/figures}
\newcommand{\tabdir}{\rootpath/tables}
\newcommand{\bibfile}{\rootpath/sources.bib}

% table alignment
\newcommand{\RR}{\raggedright\arraybackslash}
\newcommand{\RL}{\raggedleft\arraybackslash}
\newcommand{\CC}{\centering\arraybackslash}

% note for figure
\captionsetup{justification=raggedright,singlelinecheck=false}
\newcommand{\fignote}[1]{\caption*{\footnotesize {\bfseries Note.} #1}}
\newcommand{\coefplotnote}{
  \fignote{Center points represent the average estimate for each
    subgroup, with vertical lines plotting the 95\% confidence
    interval. Each point is labeled with its estimate as well as the
    $p$-value from the test of its difference from zero. Sample sizes
    for each subgroup are shown along the $x$-axis under the group
    estimate.}
}

% turn off endfloat newpage after each figure
\renewcommand{\efloatseparator}{\mbox{}}

%%%%%%%%%%%%%%%%%%%%%%%%%%%%%%%%%%%%%%%%%%%%%%%%%%%%%%%%%%%%%%%%%%%%%%%%%%%%%%%% 
% BEGIN DOCUMENT
%%%%%%%%%%%%%%%%%%%%%%%%%%%%%%%%%%%%%%%%%%%%%%%%%%%%%%%%%%%%%%%%%%%%%%%%%%%%%%%% 
\begin{document}

% header
\begin{center}
  {\sc{\bfseries Do civic returns to higher education differ
      across subpopulations? An analysis using propensity forests} \\
    Paper Tables and Figures}
\end{center}

\listoffigures
\listoftables

% ------------------------------------------------------------------------------
% FIGURES
% ------------------------------------------------------------------------------

% DESCRIPTIVE
\begin{figure}
  \caption{Variation in civic participation as seen in the data (left)
  and as a function of college participation (right).}
  \label{fig:desc}
  \includegraphics[width=\linewidth]{{\figdir/desc_het.pdf}}
  \caption*{\fignote{All values represent unweighted averages by
      subgroup. Percentages show in left-hand side facets are relative
      within enrollment condition. Histograms on the right-hand side
      show observation-level variation in the $\tau(x)$ estimates
      produced by the propensity forest models.}}
\end{figure}  

% ATE: gender, raceeth, poverty
\begin{figure}
  \caption{Estimated returns of college enrollment on voting and volunteering 
    behavior by gender, race / ethnicity, and poverty status}
  \label{fig:grp}
  \includegraphics[width=\linewidth]{{\figdir/gender_race_pov_q80.pdf}}
  \caption*{\coefplotnote}
\end{figure}

% ATE: gender X race/ethnicity
\begin{figure}
  \caption{Estimated returns of college enrollment on voting and volunteering 
    behavior: race/ethnicity by gender}
  \label{fig:g_x_r}
  \includegraphics[width=\linewidth]{{\figdir/gender_x_race_q80.pdf}}
  \caption*{\coefplotnote}
\end{figure}

% ATE: poverty status X race/ethnicity
\begin{figure}
  \caption{Estimated returns of college enrollment on voting and volunteering 
    behavior: race/ethnicity by poverty status}
  \label{fig:p_x_r}
  \includegraphics[width=\linewidth]{{\figdir/poverty_x_race_q80.pdf}}
  \caption*{\coefplotnote}
\end{figure}

% ATE: propensity
\begin{figure}
  \caption{Estimated returns of college enrollment on voting and volunteering 
    behavior by propensity of enrollment}
  \label{fig:propensity}
  \includegraphics[width=\linewidth]{{\figdir/propensity_q80.pdf}}
  \caption*{\coefplotnote}
\end{figure}

% ------------------------------------------------------------------------------
% TABLES
% ------------------------------------------------------------------------------

\processdelayedfloats
\appendix
\renewcommand\thetable{A.\arabic{table}}

\cleardoublepage
{\footnotesize
  \input{\tabdir/comparison}
}

\cleardoublepage
{\footnotesize
  \input{\tabdir/estimates}
}

\cleardoublepage
{\footnotesize
  \input{\tabdir/predlab}
}

\cleardoublepage
{\footnotesize
  \input{\tabdir/predmod}
}

\cleardoublepage
\input{\tabdir/calibration}

\cleardoublepage
{\scriptsize
  \input{\tabdir/regressions}
}

\end{document}
%%%%%%%%%%%%%%%%%%%%%%%%%%%%%%%%%%%%%%%%%%%%%%%%%%%%%%%%%%%%%%%%%%%%%%%%%%%%%%%% 
%% END DOCUMENT
%%%%%%%%%%%%%%%%%%%%%%%%%%%%%%%%%%%%%%%%%%%%%%%%%%%%%%%%%%%%%%%%%%%%%%%%%%%%%%%%

